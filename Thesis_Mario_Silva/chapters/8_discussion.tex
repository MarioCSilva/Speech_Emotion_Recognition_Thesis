\chapter{Conclusions}
\label{chapter:conc}

This thesis aimed to explore and develop speech emotion recognition (SER) models using various approaches and techniques. To achieve this goal, state-of-the-art research on emotion recognition was reviewed, and different methods were analyzed for speech emotion recognition. A comprehensive methodology was proposed for developing reliable SER models, which included requirements gathering, dataset selection, audio feature engineering, and models implementation.

The requirements gathering phase involved identifying the research objectives, selecting the appropriate dataset, and defining the evaluation criteria for the models. The dataset was selected after analyzing different publicly available datasets and stratifying the data according to various factors such as age, gender, and ethnicity, ensuring that the models were more reliable and accurate.

Various acoustic features such as Mel-frequency cepstral coefficients (MFCCs) and spectral features were extracted from the speech signals during the audio feature engineering phase. Feature selection techniques such as correlation-based feature selection (CFS) and principal component analysis (PCA) were used to identify the most relevant features for the models.

Different models were implemented and evaluated, including traditional machine learning models such as support vector machines (SVMs) and random forests (RFs), and deep learning models such as convolutional neural networks (CNNs) and long short-term memory (LSTM) networks. The models were evaluated using various performance metrics such as accuracy, precision, recall, and F1 score.

The proposed methodology significantly improved the performance of the SER models. The best model achieved an accuracy of 83.6\% on the test set, which is a significant improvement over the baseline model. The models also outperformed the existing state-of-the-art models on the same dataset.

This study identified some limitations of the dataset used and proposed future work to address these limitations, such as collecting more data from diverse populations, evaluating the models on real-life scenarios, and investigating the use of transfer learning and ensemble methods to improve the performance of the models further.

Future work in this area includes the exploration and incorporation of additional features for speech emotion recognition, such as physiological signals or facial expressions, to create more robust and accurate models. The development of models that can better handle noisy and non-stationary acoustic environments is another area of potential improvement, which could involve the exploration of novel signal processing techniques or the use of more advanced deep learning architectures.

Furthermore, there is potential for the development of emotion recognition models that are more contextually aware, such as those that can take into account situational factors or the user's history and preferences. This could enable the creation of more personalized emotion recognition services that are tailored to individual users.

Finally, integrating emotion recognition technology into a range of real-world applications, such as virtual assistants, smart homes, or healthcare monitoring systems, could be a promising future direction. This could involve the creation of new interfaces and interaction paradigms that are designed to be more emotionally intelligent and responsive to user needs.

In summary, this thesis proposes a comprehensive methodology for developing reliable SER models and demonstrates that the proposed methodology significantly improves the performance of the models. The findings of this study can be used to improve the accuracy and reliability of emotion recognition systems and contribute to the development of emotionally intelligent systems. The future of speech emotion recognition research and development is promising, with many exciting opportunities for further exploration and innovation in this area.
